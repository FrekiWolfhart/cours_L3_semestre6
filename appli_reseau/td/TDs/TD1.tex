\section{TD1}

\subsection{Exercice 1}

\subsubsection{Question 1}

L'intérêt du modèle en couche est la facilité d'abstractin et l'isolation des couches.

\subsubsection{Question 2}

Le modèle de couche TCP/IP contient 4 couches:
\begin{enumerate}
  \item Application
  \item Transport
  \item Internet
  \item NetworkAccess
\end{enumerate}

\subsubsection{Question 3}

Les deux modèles sont des modèles à couches, mais le TCP/IP en contient moins.

Couches du modèle OSI:
\begin{enumerate}
  \item Application
  \item Présentation
  \item Session
  \item Transport
  \item Network
  \item DataLink
  \item Physical
\end{enumerate}

\subsubsection{Question 4}

La couche permettant de faire abstraction de la manièrte dont les différents réseaux locaux sont interconnectés
est la couche Network pour le modèle OSI et la couche Internet pour le modèle TCP/IP.

\subsection{Exercice 2}

\subsubsection{Question 1}

Protocoles de la couche DataLink:
\begin{itemize}
  \item WiFi
  \item Ethernet
  \item Bluetooth
\end{itemize}

\subsubsection{Question 2}

Le rôle de la couche DataLink est d'assurer la communication entre plusieurs machines connectées sur un même réseau local.

\subsection{Exercice 3}

\subsubsection{Question 1}

Le protocle Ip appartient à la couche Internet.

\subsubsection{Question 2}

Afin de connaître la destination d'un pacquet, on regarde l'adresse dans la pacquet.

\subsubsection{Question 3}

La principale amélioration de L'IPv6 sur l'IPv4 est la quantité d'adresses disponibles.

\subsubsection{Question 4}

On a choisi des adresses de 128 bits pour deux choses:
\begin{enumerate}
  \item La quantité d'adresses disponibles.
  \item La généralisation des adresses à partir de l'adresse Mac.
\end{enumerate}

\subsection{Exercice 4}

$$\frac{vc}{d}<t~\Rightarrow~\frac{36}{d}<1,2~\Rightarrow~d<\frac{5\times16000}{2,4}~\Rightarrow~d<\frac{400000}{12}~\Rightarrow~d<33333,333~m$$

Cette solution prends en compte le retour du disque vers le point de départ.

Si on compte un allé simple, on fait $33,333\times2=66,67km,$ ou bien 66666,67 m.
