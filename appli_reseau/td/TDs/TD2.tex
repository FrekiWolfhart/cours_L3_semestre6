\section{TD2}

\subsection{Exercice 1}

\subsubsection{Question 1}

Afin de représenter l'adresse de ce serveur, on utilise la classe InetAddress de Java.

\subsubsection{Question 2}

\begin{lstlisting}[language=Java]
  public static InetAddress getInetAddressByName(String hostname){
    return InetAddress.getByName(hostname);
  }
\end{lstlisting}

\subsection{Exercice 2}

\subsubsection{Question 1}

Les classes importantes pour UDP sont DatagramSocket et DatagramPacket.

\subsubsection{Question 2}

Afin de créer un DatagramPacket, un client à besoin de:
\begin{itemize}
  \item Le buffer.
  \item La taille du buffer.
  \item L'adresse du serveur.
  \item Le port d'écriture.
\end{itemize}

\subsubsection{Question 3}

\begin{lstlisting}[language=Java]
  public static void receiveTextFromUdp(){
    DatagramSocket s = new DatagramSocket(ssss);
    byte msg[]= new byte[10];
    DatagramPacket n = new DatagramPacket(msg, 40);
    s.receive(in);
    System.out.println(new String(msg));
    s.close();
  }
\end{lstlisting}

\subsection{Exercice 3}

\subsubsection{Question 1}

La différence entre les transmission UDP et TCP est que UDP envoie un flux unidirectionnel et plus rapides, mais a des pertes,
alors que le TCP est plus lent mais n'a aucune perte.

\subsubsection{Question 2}

Les données UDP se transmettent plus vite.

\subsubsection{Question 3}

\begin{lstlisting}
  Socket s = new Socket(adresse, port);
  ObjectOutputStream output = new ObjectOutputStream(s.getOutputStream);
  ObjectInputStream input = new ObjectInputStream(s.getInputStream);
  input.read();
  output.write();
  s.close();
\end{lstlisting}

\subsubsection{Question 4}

\begin{lstlisting}[language=Java]
  public static void receiveMessageTCP(){
    Socket s = new Socket(myServer, ssss);
    InputStream in = s.getInputStream();
    BufferedReader b = new BufferedReader(new InputStreamReader(in));
    System.out.println(b.readline());
    s.close();
  }
\end{lstlisting}
