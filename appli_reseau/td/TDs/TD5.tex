\section{TD 5}

\subsection{Exercice 1}

\subsubsection{Question 1}

Afin d'assurer la communication non-bloquante pour la
lecture ou l'\'ecriture dans un SocketChannel en Java NIO
il faut utiliser\\ la m\'ethode
channel.configureBlocking(false).

Les objets de type SelectableChannels ont comme
propri\'et\'es:
\begin{itemize}
  \item Avoir une configuration bloquante ou non bloquante.
  \item On peux l'enregistrer dans un Selector.
\end{itemize}


\subsubsection{Question 2}

2 octets seront transf\'er\'es vers socketChannel.

Les nouvelles valeurs seront:
\begin{itemize}
  \item \emph{Position} = 3
  \item \emph{Limit} = 3
  \item \emph{Capacity} = 8
\end{itemize}


\subsubsection{Question 3}

Les diff\'erences entre Buffer.clear() et Buffer.flip() sont:
\begin{itemize}
  \item Buffer.clear() remet la position \`a 0.
  \item Buffer.flip() remet la position \`a gauche des
  donn\'ees qu'on a \'ecrit, et la limite \`a droite des
  donn\'ees afin de ne pouvoir travailler que sur des
  donn\'ees qu'on \`a d\'ej\`a \'ecrites.
\end{itemize}


\subsection{Exercice 2}

\subsubsection{Question 1}

La classe Selector est un classe qui donne des informations
sur des Channel sur lesquels on veut travailler, via
les SelectionKey.

La classe SelectionKey est une interface entre le Selector
et les Channel, qui permet de au Selector de
r\'ecup\'erer des informations sur le Channel.

La question est mal pos\'ee, on demande quel quantit\'e de
types plut\^ot qu'une quantit\'e de cl\'es, et on peux en
associer 4 \`a un Selector.

\subsubsection{Question 2}

On peux enregistrer un SocketChannel \`a un Selector pour
attendre les \'ev\'enements de type \'ecriture en utilisant
la m\'ethode\\socketChannel.register(selector,
SelectionKey.OP\_WRITE).

On peux enregistrer un ServerSocketChannel \`a un
Selector pour attendre les demandes de connection de clients
avec la m\'ethode\\
serverSocketChannel.register(selector,
SelectionKey.OP\_ACCEPT).

\subsubsection{Question 3}

\begin{lstlisting}[language=Java,caption={Selector
    qui attend des demandes de connection \`a un serveur
    et des demandes d'\'ecriture d'un client},captionpos=b]
  // Java NIO cree des objets avec open() au lieu de new.
  Selector selector = Selector.open();
  // Il ne faut pas oublier de mettre les channels non bloquants.
  ssc.configureBlocking(false);
  csc.configureBlocking(false);
  ssc.register(selector, SelectionKey.OP_ACCEPT);
  csc.register(selector, SelectionKey.OP_WRITE);
  while(true){
    selector.select();
    Iterator<Set<SelectionKey> > iterator = selector.selectedKey().iterator();
    while(iterator.hasNext()){
      SelectionKey key = iterator.next();
      if(key.isAcceptable()){
        scs.accept();
      }
      if(key.isWriteable()){
        // ecrire
      }
      iterator.remove();
    }
  }
\end{lstlisting}

\subsection{Exercice 3}

\subsubsection{Question 1}

Il est possible d'utiliser un seul Selector partag\'e par
plusieurs threads.

La m\'ethode select(\`) est prot\'eg\'ee, alors que la
gestion des cl\'es non. Il faut donc toujours v\'erifier
les cl\'es lors du multi-threading.

\subsubsection{Question 2}

La m\'ethode Selector.wakeup() sert \`a sort un thread
de la s\'election lorsqu'il est bloqu\'e.

On a besoin de l'utiliser quand on a besoin de modifier
le selector.

\subsubsection{Question 3}

\begin{enumerate}
  \item Communication d'un channel vers plusieurs buffers:
  \begin{itemize}
    \item SocketChannel.write(ByteBuffer[] srcs)
    \item ScatteringByteChannel.write(ByteBuffer[] srcs)(interface)
  \end{itemize}
  \item Communication de plusieurs buffers vers un channel:
  \begin{itemize}
    \item SocketChannel.read(ByteBuffer[] dsts)
    \item GatheringByteChannel.read(ByteBuffer[] dsts)(interface)
  \end{itemize}
\end{enumerate}

Les deux interfaces impl\'ementent les m\'ethodes dans
le SocketChannel.
