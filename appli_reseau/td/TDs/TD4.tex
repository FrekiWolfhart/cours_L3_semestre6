\section{TD 4}

\subsection{Exercice 1}

\subsubsection{Question 1}

Afin d'utiliser les threads en Java, il faut créer une classe que étend la classe Thread.
La méthode principale pour un thread est start.

\subsubsection{Question 2}

\begin{lstlisting}[language=Java,caption={code de la classe SocketHandler},captionpos=b]
  public class SocketHandler implements Runnable{
    private Socket socket;
    private List<String> messages;

    public SocketHandler(Socket socket){
      this.socket = socket;
      messages = new ArrayList<String>();
    }

    // TODO: Ajouter la gestion d'exceptions
    @Override
    void run(){
      BufferedReader input = new BufferedReader(new InputStreamReader(this.socket.getInputStream());
      String line;
      while((line = input.readLine()) != null) this.messages.add(line);

      socket.close();
    }

    public Socket getSocket(){
      return this.socket;
    }

    public List<String> getMessages(){
      return this.messages;
    }

    public void setSocket(Socket socket){
      this.socket = socket;
    }
  }
\end{lstlisting}

\subsection{Exercice 2}

\subsubsection{Question 1}

On peux lui faire créer un thread par client TCP lu.

\begin{lstlisting}[language=Java,caption={Serveur TCP multi-thread en trois lignes},captionpos=b]
  ServerSocket server = new ServerSocket(1234);
  while(true)
    new Thread(new SocketHandler(server.accept())).start();
\end{lstlisting}

\subsubsection{Question 2}

Comportement eronné(duplication potentielles) et imprévisible.

\subsubsection{Question 3}

La structure ne sera modifiée que par un seul à la fois via un système de sémaphores et de section critiques, si la structure est sérialisée.

\begin{lstlisting}[language=Java,caption={Exemple d'écriture sur la même structure de données par deux threads},captionpos=b]
  class Data{
    public int data=0;
  }

  class Job implements Runnable{
    Data d;

    Job(Data d){
      this.data = data;
    }

    void run(){
      int newData = d.data+1;
      d.data = newData;
    }
  }

  public static void main(String[] args){
    Data d = new Data();

    Thread A = new Thread(new Job(d).start());
    Thread B = new Thread(new Job(d).start());

    System.out.println(d.data);
  }
\end{lstlisting}

La donnée affichée sera soit 1 soit 2.

\subsubsection{Question 4}

Il n'y a aucun moyen.

\subsection{Exercice 3}

\subsubsection{Question 1}

Pour la communication entre un socket server et un socket client, il faut utiliser les classes ServerSocketChannel et SocketChannel de java.nio.channels.

\subsubsection{Question 2}

Utiliser Java NIO pour la communication entre sockets a comme avantages:
\begin{itemize}
  \item non-bloquant
  \item pas de threads
  \item plus optimisé
\end{itemize}

Il est globalement très efficace.

\subsubsection{Question 3}

On utilise un Selector pour les communications entre sockets car il trouve ce qu'il faut faire dans l'instant.
C'est-à-dire, il dit quel channel est prêt à travailler, et comment.

\begin{lstlisting}[language=Java,caption={Exemple selector},captionpos=b]
  selector.select();
  Set<SelectionKey> selectedKeys = selector.selectedKeys();
  Iterator<SelectionKey> iterator = selectedKeys.iterator();
  while(iterator.hasNext()){
    SelectionKey key = iterator.next();
  }
\end{lstlisting}

\begin{figure}[h!]
  \centering
    \begin{tikzpicture}
      \node[state] at (0,0)   (0){SelectionKey};
      \node[state] at (-2,-2) (1){Channel};
      \node[state] at (2,-2)  (2){Selector};
      \draw[->] (2) edge[right] node{select()}  (0)
                (0) edge[left]  node{channel()} (1)
                (1) edge[below] node{selector.register($\begin{matrix} channel,\\ SelectionKey.OP\_READ,\\ SelectionKey.OP\_WRITE\\
                                                      SelectionKey.OP\_CONNECT\\ SelectionKey.OP\_ACCEPT \end{matrix}$)} (2);
    \end{tikzpicture}
  \caption{Schéma classes}
\end{figure}

Les clés en majuscules sont à choisir en fonction de ce dont on a besoin.

\subsubsection{Question 4}

\begin{lstlisting}[language=Java,caption={Serveur Mono-Thread qui gère jusqu'à 10 clients en simultanés},captionpos=b]
  Selector selector = Selector.open();
  ServerSocketChannel server = ServerSocketChannel().open;
  server.configureBlocking(false);
  selector.register(server, SelectionKey.OP_ACCEPT);
  // Exemple selector, code ensuite dans la boucle while
  if(key.isAcceptable()){
    SocketChannel client = server.accept();
    selector.register(client, (SelectionKey.OP_WRITE|SelectionKey.OP_READ));
    client.configureBlocking(false);
  }
  if(key.isReadable()){
    SocketChannel client = key.channel();
    // lecture
  }
\end{lstlisting}
