\documentclass[12pt, a4paper, draft]{article}
\usepackage[T1]{fontenc}
\usepackage[french]{babel}
\usepackage[utf8]{inputenc}
\usepackage{lmodern}
\usepackage{amsmath}
\usepackage{amssymb}
\usepackage{algorithm}
\usepackage{algorithmic}
\usepackage{enumitem}
\usepackage{multirow}

\title{Notes de cours de calculabilité}
\author{Yann Miguel}

\begin{document}
\ttfamily
\maketitle
\tableofcontents
\newpage

\section{Cours 1}
Un problème peut être simple à définir, mais ne pas admettre\\une solution algorithmique.
Par exemple, la suite de Collatz.

\begin{algorithm}
  \caption{Collatz(n)}
  \begin{algorithmic}
    \STATE print n
    \IF{n=1}
      \STATE return 0
    \ELSE
      \IF{n=0(mod2)}
        \STATE Collatz(n/2)
      \ELSE
        \STATE Collatz(3n+1)
      \ENDIF
    \ENDIF
  \end{algorithmic}
\end{algorithm}

\subsection{Machine de Turing}
Il s'agit d'un modèle mathématique abstrait.\\
\textbf{Définition:}\\
Une machine de Turing(MT) déterministe est un $\gamma$-uplet
$$M~=~(Q,\Sigma,\Gamma,\delta,q_0,B,q_F)$$
où:
\begin{itemize}
  \item Q est un ensemble d'états fini
  \item $\Sigma$ est l'alphabet d'entrée
  \item $\Gamma$ est l'alphabet de ruban\\(contient tout les
  symboles pouvant apparaître sur le\\ruban)
  \item $\delta$ est une fonction de transition
  \item q$_0\in$Q est l'état initial
  \item B $\in\Gamma\backslash\Sigma$est un symbole blanc spécial\\(ne fait pas
  parti de l'alphabet d'entrée)
  \item q$_F\in$Q est l'état final
\end{itemize}

Finalisations:\\
\begin{tabular}{|c|c|}
  \hline
  Résultat & Condition\\
  \hline
  acceptation & la machine entre et s'arrête sur l'état final\\
  \hline
  \multirow{2}{4em}{rejet} & s'arrête dans un état non final\\
  & la machine ne s'arrête pas\\
  \hline
\end{tabular}

\textbf{Définition:}\\
Une description instantanée(DI) d'une MT décrit sa\\configuration courante.
C'est un mot
$$uqav~\in~(\{\epsilon\}\cup(\Gamma\backslash\{B\})\Gamma^*)Q\Gamma(\{\epsilon\}\cup\Gamma^*(\Gamma\backslash\{B\}))$$
avec:
\begin{itemize}
  \item Q, l'état courant
  \item u, v $\in\Gamma^*$, le contenu du ruban à gauche et à droite de la
  \\tête, jusqu'au dernier symbole non blanc
  \item a $\in\Gamma$, le symbole de ruban actuellement sous la tête
\end{itemize}

\textbf{Définition:}\\
Un mouvement, une transition, un déplacememnt de la MT à partir\\de la DI
$\alpha$ = uqav vers la DI suivante $\beta$ sera noté $\alpha\vdash\beta$.\\
Plus précisement:
\begin{enumerate}
  \item Si $\delta$(q, a) = (p, b, L),
  \begin{itemize}
    \item si u = $\epsilon$, alors $\beta$ = pBbv
    \\(potentiellement en supprimantles B à la fin de bv)
    \item si u = u'c avec c $\in\Gamma$ alors $\beta$ = u'pcbv
    \\(potentiellement en supprimant les B à la fin de bv)
  \end{itemize}
  \item si $\delta$(q, a) = (p, b, R)
  \begin{itemize}
    \item si v = $\epsilon$ alors $\beta$ = ubpB
    \\(potentiellement en supprimant les B au début de ub)
    \item si v $\neq\epsilon$ alors $\beta$ = ubpv
    \\(potentiellement en supprimant les B au début de ub)
  \end{itemize}
  \item si $\delta$(q, a) est indéfini, alors
  \\aucun mouvement n'est possible depuis $\alpha$ et $\alpha$ est une DI\\d'arrêt.
  Si q = q$_F$, alors $\alpha$ est une DI acceptante.
\end{enumerate}

\newpage

\section{Contact/MCC}

\begin{tabular}{|c|c|}
  \hline
  Type & Lien\\
  \hline
  site & aeporreca.org/calculabilite(ou le lien discord)\\
  mail & antonio.porreca@lis-lab.fr\\
  note & 0.2*CC+0.8*ET\\
  \hline
\end{tabular}

\end{document}
