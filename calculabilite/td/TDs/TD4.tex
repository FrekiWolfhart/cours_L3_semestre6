\section{TD4}

\subsection{Exercice 1}

\subsubsection{Question 1}

Famille des langages non r\'ecursifs.

\begin{itemize}
  \item L$_1$=\{$\langle\ M\ \rangle \!\ $
  | M accepte a\} est un langage
  non r\'ecursif.
  \item L$_2$=\{$\langle\ M\ \rangle \!\ $
  | M n'accepte pas a.\} est un langage 
  non r\'ecursif.
\end{itemize}

Cl\^oture par intersection:

Or, l'intersection de L$_1$ et L$_2$
est vide, et le langage vide est
r\'ecursif.\\
Donc, l'ensemble des langages
non r\'ecursifs n'est pas clos par
intersection.\\\\

Cl\^oture par union:

Or, l'union de L$_1$ et L$_2$ est
n'importe quel code de machine, qui est
r\'ecursif.\\
Donc, l'ensemble des langages
non r\'ecursifs n'est pas clos par
union.

\subsubsection{Question 2}

Famille des langages
non r\'ecursivement \'enum\'erables.\\
Pour prouver un langage non
r\'ecursivement \'enum\'erable,
on peux faire une r\'eduction
vers L$_{\bar{u}}$, ou utiliser le
th\'eor\`eme de Rice.\\\\

Cl\^oture par compl\'ementation:

Le compl\'ementaire de L$_u$ est
n'est r\'ecursivement \'enum\'erable pas,
alors que L$_u$ l'est.\\
Donc, l'ensemble des langages
non r\'ecursivement \'enum\'erables
n'est pas clos par compl\'ementation.\\\\

Cl\^oture par intersection:

\begin{itemize}
  \item L$_1$=\{$\langle\ M\ \rangle \!\ $
  \# 0 | a $\not\in$ L(M)\} est
  non r\'ecursivement \'enum\'erable.
  \item L$_2$=\{$\langle\ M\ \rangle \!\ $
  \# 1 | M n'accepte pas a\} est
  non r\'ecursivement \'enum\'erable.
\end{itemize}

L'intersection de L$_1$ et de L$_2$ est
vide, et le langage vide est r\'ecursif.
Donc, l'ensemble des langages
non r\'ecursivement \'enum\'erables
n'est pas clos par intersection.\\\\

Cl\^oture par union:

\begin{itemize}
  \item L$_1$=\{$\langle\ M\ \rangle \!\ $
  | a $\not\in$ L(M)\} est
  non r\'ecursivement \'enum\'erable.
  \item L$_2$=\{$\langle\ M\ \rangle \!\ $
  | a $\in$ L(M) ou b $\not\in$ L(M)\} est
  non r\'ecursivement \'enum\'erable.
\end{itemize}

L'union de L$_1$ et de L$_2$ est
l'ensemble des codes de machines, qui est
r\'ecursif.\\
Donc, l'ensemble des langages
non r\'ecursivement \'enum\'erables
n'est pas clos par union.

\subsubsection{Question 3}

Famille des langages r\'ecursifs.

Donc, $\forall$ L non r\'ecursif
$\Rightarrow~\bar{L}$ non r\'ecursif.

Or, $\forall$ L, L r\'ecursif
$\Rightarrow~\bar{L}$ r\'ecursif.

Donc, par contrapos\'ee,
$\forall$ L, $\bar{L}$ non recursif
$\Rightarrow$ L non recursif.

On choist comme langage L $\bar{L}$, et donc,
la contrapos\'ee devient:\\
$\forall$ L non r\'ecursif
$\Rightarrow~\bar{L}$ non r\'ecursif.

\subsection{Exercice 2}

On ne peux utiliser le th\'eor\`eme de
Rice que sur une propri\'et\'e
non triviale.

\subsubsection{Question 1}

\{L | L=a$^*$\} doit \^etre montr\'ee
non trivial.

Afin de le montrer non trivial, on doit
trouver deux machines de Turing,
une qui accepte la propri\'et\'e et une
qui la refuse.
\begin{enumerate}
  \item M$_1$, la machine qui boucle sur
  des a jusqu'\`a la fin du mot,
  accepte la propri\'et\'e.
  \item M$_2$, la machine qui refuse tout,
  refuse la propri\'et\'e.
\end{enumerate}

Par cons\'equent, la propri\'et\'e est
non-triviale.\\
Donc, on ne peux pas d\'ecider si le
langage d'une machine est a$^*$, et donc
, d'apr\`es le th\'eor\`eme de Rice,
la propri\'et\'e n'est pas r\'ecursive.

\subsubsection{Question 2}

\{L | aa $\in$ L, et
$\forall$k$\neq$2:a$^k\not\in$L\} doit
\^etre montr\'ee non triviale.

On doit donc trouver deux machines de
Turing, une qui accepte et une qui refuse
la propri\'et\'e.
\begin{enumerate}
  \item M$_1$, la machine qui n'accepte
  que un a suivi d'un a suivi d'un blanc
  accepte la propri\'et\'e.
  \item M$_2$, la machine qui refuse tout,
  refuse la propri\'et\'e.
\end{enumerate}

Par cons\'equent, la propri\'et\'e est
non-triviale.\\
Donc, on ne peux pas d\'ecider si le
langage d'une machine est aa, et donc
, d'apr\`es le th\'eor\`eme de Rice,
la propri\'et\'e n'est pas r\'ecursive.

\subsubsection{Question 3}

\{L | ab $\not\in$ L, ou $\exists$k :
ab$^k\in$ L\} doit \^etre montr\'ee
non triviale.

Or, cette propri\'et\'e contient tout
les langages, \'etant donn\'e
que soit elle contient ab, soit elle ne
contient pas ab, et que tout les langages
soit contiennent ab, soit ne contiennent
pas ab.\\
Par cons\'equent, elle est triviale
pour toute machine M.

\subsubsection{Question 4}
\{\{$\langle\ M\ \rangle \!\ $
| M n'accepte pas
$\langle\ M\ \rangle \!\ $\}\}
doit \^etre montr\'ee non triviale.

On peux facilement trouver une machine
qui r\'efute cette propri\'et\'e, mais,
si on essaye de trouver une machine
qui accepte cette propri\'et\'e,
on tombe sur une contradiction,
peu importe si le langage de la machine
en question est dans ou hors du langage.\\
Donc, cette propri\'et\'e est triviale.

\subsubsection{Question 5}
\{\{$\langle\ M\ \rangle \!\ $
| M accepte $\langle\ M\ \rangle \!\ $\}\}
doit \^etre montr\'ee non triviale.
