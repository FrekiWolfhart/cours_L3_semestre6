\section{TD3}

\subsection{Exercice 1}

\subsubsection{Question 1}

L'énoncé est vrai car le théorème de l'arrêt dit exactement cela.

\subsubsection{Question 2}

L'énoncé est faux, car il existe une machine, la machine qui accepte tout, qui va tout le temps s'arrêter, peu impote les entrées.\\
Si les entrées s'arrêtent, on prends la machine qui s'arrête tout le temps. Sinon, on prends celle qui ne s'arrête jamais.\\
Par conséquent, $\forall \langle\ M\ \rangle \!\ ,w~\exists~M_{halt}(\langle\ M\ \rangle \!\ , w)~=~halt(\langle\ M\ \rangle \!\ , w)$

\begin{figure}[ht]
  \centering
  \begin{tikzpicture}
    \node[state, initial, accepting] (0){acc};
  \end{tikzpicture}
  \caption{Machine qui accepte tout}
\end{figure}

\begin{figure}[ht]
  \centering
  \begin{tikzpicture}
    \node[state, initial] (0){ref};
  \end{tikzpicture}
  \caption{Machine qui refuse tout}
\end{figure}

\subsection{Exercice 2}

La réduction est une fonction $f:~\langle\ M\ \rangle \!\ \rightarrow f(\langle\ M\ \rangle \!\ )$.

\subsubsection{Question 1}

g(M) est une machine qui supprime aa puis lance la machine M.

Réduction:
\begin{itemize}
  \item Si M s'arrête sur 0, g(M) commence par supprimer aa, puis fait tourner M sur ce qu'il reste,\\
  c'est-à-dire 0, et donc g(M) termine sur aa.
  \item g(M) s'arrête sur aa. Hors, g(M) commence par supprimer aa puis lance la machine M, qui s'arrête sur 0.\\
  Donc, si g(M) s'arrête sur aa, M va s'arrêter sur 0, qui est le reste.
\end{itemize}

Donc, par définition de f, $\langle\ M\ \rangle \!\ \in L_{halte} \leftrightarrow f(\langle\ M\ \rangle \!\ )\in A$.

Le langage L$_{halte}$ n'étant pas récursif, sa réduction ne l'est pas non plus.\\
Cette réduction pouvant se faire dans l'autre sens, les deux problèmes sont de difficulté équivalentes.


\subsubsection{Question 2}

f(M) est une fonction qui prends $\langle\ M\ \rangle \!\ $ et un mot w et qui donne $\langle\ M'\ \rangle \!\ $.\\
Cette fonction dit que M' accepte a si et seulement si M accepte w.

On veut donc que la machine M', machine associée à f($\langle\ M\ \rangle \!\ $), supprime a, écrive w, retourne à gauche du mot, et lance M.

Réduction:
\begin{itemize}
  \item M accepte w par définition de L$_u$.\\
  Si on lance la machine M' sur a,\\
  alors elle supprime le a, écrit le w, retourne à gauche du mot, et accepte w.\\
  Par conséquent, la machine M' a accepté l'entrée a, et f($\langle\ M\ \rangle \!\ $ \# w) $\in$ B.
  \item Si f($\langle\ M\ \rangle \!\ $ \# w) $\in$ B, alors la machine M' accepte a.\\
  Or, M' efface a, écrit w et lance M sur w, ce qui veut dire que M accepte w.\\
  Donc, $\langle\ M\ \rangle \!\ $ \# w$\in$L$_u$.
\end{itemize}

\subsubsection{Question 3}

f(M) est une fonction qui prends $\langle\ M\ \rangle \!\ $ et un mot w et qui donne $\langle\ M'\ \rangle \!\ $.\\
Cette fonction dit que  M' refuse w mais accepte bbw si et seulement si M refuse w.

On veut donc que la machine M' puisse vérifier si son entrée est bbw.\\
Si c'est le cas, elle l'accepte. Sinon, elle lance M.

Réduction:
\begin{itemize}
  \item Si $\langle\ M\ \rangle \!\ $ \# w $\in$ L$_{\bar{u}}$, alors M n'accepte pas w.\\
  Posons $\langle\ M'\ \rangle \!\ $ \# w = f($\langle\ M\ \rangle \!\ $), w $\neq$ bbw,\\
  alors si on lance M' sur w, elle lance M sur w, et n'accepte pas.\\
  De plus, par définition, de M', M' accepte bbw. Donc, f($\langle\ M\ \rangle \!\ $) $\in$ C.
  \item Si $\langle\ M'\ \rangle \!\ $ \# w = f($\langle\ M\ \rangle \!\ $) $\in$ C.
  \begin{itemize}
    \item M' accepte bbw
    \item M' n'accepte pas w
    \item Donc, M' sur w lance M sur w, donc M n'accepte pas w.
  \end{itemize}
  Donc, $\langle\ M\ \rangle \!\ $ \# w $\in$ L$_{\bar{u}}$.
\end{itemize}


\subsubsection{Question 4}

f(M) est une fonction qui prends un mot w et qui retourne le même mot précédé par un a.
Cette fonction dit que M' accepte aw si et seulement si M accepte w.

Il suffit juste que M' prenne un mot w accepté par M, lui rajoute un a en première lettre, et accepte le mot.

\begin{itemize}
  \item Si L est récursif, on ne peux rien en déduire.
  \item Si aL est récursif, alors L est récursif.
  \item Si L n'est pas récursif, alors aL ne l'est pas.
  \item Si L est récursivement énumérable, alors on ne peux rien en déduire.
  \item Si aL est récursivement énumérable, alors L est récursivement énumérable.
  \item Si L n'est pas récursivement énumérable, alors aL ne l'est pas.
\end{itemize}

\subsubsection{Question 5}
