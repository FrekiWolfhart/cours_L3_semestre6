\section{TD3}

\subsection{Exercice 1}

\subsubsection{Question 1}

L'énoncé est vrai car le théorème de l'arrêt dit exactement cela.

\subsubsection{Question 2}

L'énoncé est faux, car il existe une machine, la machine qui accepte tout, qui va tout le temps s'arrêter, peu impote les entrées.\\
Si les entrées s'arrêtent, on prends la machine qui s'arrête tout le temps. Sinon, on prends celle qui ne s'arrête jamais.\\
Par conséquent, $\forall \langle\ M\ \rangle \!\ ,w~\exists~M_{halt}(\langle\ M\ \rangle \!\ , w)~=~halt(\langle\ M\ \rangle \!\ , w)$

\begin{figure}[ht]
  \centering
  \begin{tikzpicture}
    \node[state, initial, accepting] (0){acc};
  \end{tikzpicture}
  \caption{Machine qui accepte tout}
\end{figure}

\begin{figure}[ht]
  \centering
  \begin{tikzpicture}
    \node[state, initial] (0){ref};
  \end{tikzpicture}
  \caption{Machine qui refuse tout/tourne en boucle à l'infini}
\end{figure}

\subsection{Exercice 2}

La réduction est une fonction $f:~\langle\ M\ \rangle \!\ \rightarrow f(\langle\ M\ \rangle \!\ )$.

\subsubsection{Question 1}

g(M) est une machine qui supprime aa puis lance la machine M.

Réduction:
\begin{itemize}
  \item Si M s'arrête sur 0, g(M) commence par supprimer aa, puis fait tourner M sur ce qu'il reste,\\
  c'est-à-dire 0, et donc g(M) termine sur aa.
  \item g(M) s'arrête sur aa. Hors, g(M) commence par supprimer aa puis lance la machine M, qui s'arrête sur 0.\\
  Donc, si g(M) s'arrête sur aa, M va s'arrêter sur 0, qui est le reste.
\end{itemize}

Donc, par définition de f, $\langle\ M\ \rangle \!\ \in L_{halte} \leftrightarrow f(\langle\ M\ \rangle \!\ )\in A$.

Le langage L$_{halte}$ n'étant pas récursif, sa réduction ne l'est pas non plus.\\
Cette réduction pouvant se faire dans l'autre sens, les deux problèmes sont de difficulté équivalentes.


\subsubsection{Question 2}

f(M) est une fonction qui prends $\langle\ M\ \rangle \!\ $ et un mot w et qui donne $\langle\ M'\ \rangle \!\ $.\\
Cette fonction dit que M' accepte a si et seulement si M accepte w.

On veut donc que la machine M', machine associée à f($\langle\ M\ \rangle \!\ $), supprime a, écrive w, retourne à gauche du mot, et lance M.

Réduction:
\begin{itemize}
  \item M accepte w par définition de L$_u$.\\
  Si on lance la machine M' sur a,\\
  alors elle supprime le a, écrit le w, retourne à gauche du mot, et accepte w.\\
  Par conséquent, la machine M' a accepté l'entrée a, et f($\langle\ M\ \rangle \!\ $ \# w) $\in$ B.
  \item Si f($\langle\ M\ \rangle \!\ $ \# w) $\in$ B, alors la machine M' accepte a.\\
  Or, M' efface a, écrit w et lance M sur w, ce qui veut dire que M accepte w.\\
  Donc, $\langle\ M\ \rangle \!\ $ \# w$\in$L$_u$.
\end{itemize}

\subsubsection{Question 3}

f(M) est une fonction qui prends $\langle\ M\ \rangle \!\ $ et un mot w et qui donne $\langle\ M'\ \rangle \!\ $.\\
Cette fonction dit que  M' refuse w mais accepte bbw si et seulement si M refuse w.

On veut donc que la machine M' puisse vérifier si son entrée est bbw.\\
Si c'est le cas, elle l'accepte. Sinon, elle lance M.

Réduction:
\begin{itemize}
  \item Si $\langle\ M\ \rangle \!\ $ \# w $\in$ L$_{\bar{u}}$, alors M n'accepte pas w.\\
  Posons $\langle\ M'\ \rangle \!\ $ \# w = f($\langle\ M\ \rangle \!\ $), w $\neq$ bbw,\\
  alors si on lance M' sur w, elle lance M sur w, et n'accepte pas.\\
  De plus, par définition, de M', M' accepte bbw. Donc, f($\langle\ M\ \rangle \!\ $) $\in$ C.
  \item Si $\langle\ M'\ \rangle \!\ $ \# w = f($\langle\ M\ \rangle \!\ $) $\in$ C.
  \begin{itemize}
    \item M' accepte bbw
    \item M' n'accepte pas w
    \item Donc, M' sur w lance M sur w, donc M n'accepte pas w.
  \end{itemize}
  Donc, $\langle\ M\ \rangle \!\ $ \# w $\in$ L$_{\bar{u}}$.
\end{itemize}


\subsubsection{Question 4}

f(M) est une fonction qui prends un mot w et qui retourne le même mot précédé par un a.
Cette fonction dit que M' accepte aw si et seulement si M accepte w.

Il suffit juste que M' prenne un mot w accepté par M, lui rajoute un a en première lettre, et accepte le mot.

\begin{itemize}
  \item Si L est récursif, on ne peux rien en déduire.
  \item Si aL est récursif, alors L est récursif.
  \item Si L n'est pas récursif, alors aL ne l'est pas.
  \item Si L est récursivement énumérable, alors on ne peux rien en déduire.
  \item Si aL est récursivement énumérable, alors L est récursivement énumérable.
  \item Si L n'est pas récursivement énumérable, alors aL ne l'est pas.
\end{itemize}

\subsubsection{Question 5}

f(M) est une fonction qui prends un mot aw, et qui retourne le mot w, $\forall$w.

Supposons $\exists$ u$\not\in$L, et f(w)=u si w ne commences pas par a.
\begin{enumerate}
  \item Montrer que, si w$\in$aL$\Rightarrow$f(w)$\in$L.
  \begin{itemize}
    \item Soit w$\in$aL$\Rightarrow$w=av, avec v$\in$L
    \item donc f(w)=v et f(w)$\in$L.
  \end{itemize}
  \item Montrer que f(w)$\in$L$\Rightarrow$w$\in$aL.
  \begin{itemize}
    \item Soit w tel que f(w)$\in$L, alors f(w)$\neq$u
    \item et w commence donc par a, c'est-à-dire w=av, donc f(w)=v
    \item et v$\in$L donc av$\in$aL.
  \end{itemize}
\end{enumerate}

Donc, on a bien une réductione de aL vers L, donc:
\begin{itemize}
  \item Si L est récursif, alors aL l'est aussi.
  \item Si aL est récursif, alors L l'est aussi(question précédente).
  \item Si L est récursivement énumérable, alors aL l'est aussi.
  \item Si aL est récursivement énumérable, alors L l'est aussi(question précédente).
\end{itemize}

\subsubsection{Question 6}

M$_1$ est la machine de Turing qui s'arrête tout le temps.
M$_2$ est la machine de Turing qui tourne en boucle à L'infini.

$\langle\ M_1\ \rangle \!\ \# \in L_u. \langle\ M_2\ \rangle \!\ \# \not\in L_u.$

f est la fonction telle que f(a)=$\langle\ M_1\ \rangle \!\ $ \# et $\forall$ v$\neq$a, f(v)=$\langle\ M_2\ \rangle \!\ \# $.

\begin{enumerate}
  \item Montrer que w$\in$L$_{stupide}\Rightarrow$f(w)$\in$L$_u$.
  \begin{itemize}
    \item Soit w$\in$L$_{stupide}\Rightarrow$w=a.
    \item f(w)=$\langle\ M_1\ \rangle \!\ $ \# $\Rightarrow$ f(w)$\in$L$_u$.
  \end{itemize}
  \item Montreer que f(w)$\in$L$_u\Rightarrow$w$\in$L$_{stupide}$
  \begin{itemize}
    \item Soit w tel que f(w)$\in$L$_u\Rightarrow$f(w)$\neq\langle\ M_2\ \rangle \!\ $ \#.
    \item $\Rightarrow$f(w)=$\langle\ M_1\ \rangle \!\ $ \# donc w=a.
    \item Donc, w$\in$L$_{stupide}$.
  \end{itemize}
\end{enumerate}

\subsection{Exercice 3}

\subsubsection{Question 1}

D = \{$\langle\ M\ \rangle \!\ $|M s'arrête sur ab et ba\}

\begin{enumerate}
  \item Créer une réduction entre ce langage et un non décidable.
  \item Utiliser le théorème de Rice.
\end{enumerate}

Cette propriété étant non-triviale, d'après le théorème de Rice, ce langage n'est pas décidable.
Or, dans cet exercice, on veut faire des réductions. Donc, on doit trouver une réduction.

L$_{halte}$ = \{$\langle\ M\ \rangle \!\ $|M s'arrête sur l'entrée vide\} n'est pas décidable.

Donc, si on réduit L$_{halte}$ à D, on prouvera que D n'est pas décidable.

On doit donc trouver une fonction qui va de L$_{halte}$ vers D.

Soit f la fonction telle que f($\langle\ M\ \rangle \!\ $) = $\langle\ M'\ \rangle \!\ $ où M' est la machine qui:
\begin{itemize}
  \item sur ab accepte
  \item sur ba efface ba, et lance M
  \item sur le reste refuse
\end{itemize}

De plus, $\forall$ w qui n'est pas un code de MT, f(w)=w.

\begin{enumerate}
  \item Montrer que w$\in$L$_{halte}\Rightarrow$f(w)$\in$D.
  \begin{itemize}
    \item soit w$\in$L$_{halte}\Rightarrow$w=$\langle\ M\ \rangle \!\ \in$L$_{halte}$.
    \item donc, f(w) est un code de machine, et la machine associée a des propriétés similaires à M'.
    \item Donc, f(w)$\in$D.
  \end{itemize}
  \item Montrer que $\forall$w, f(w)$\in$D$\Rightarrow$w$\Rightarrow$L$_{halte}$
  \begin{itemize}
    \item f(w)=$\langle\ M'\ \rangle \!\ \Rightarrow$ w = $\langle\ M\ \rangle \!\ $
    \item comme $\langle\ M'\ \rangle \!\ $ donc M' s'arrête sur ba.
    \item Mais, sur ba, M' efface ba et lance M
    \item donc M s'arrête sur l'entrée vide
    \item donc w = $\langle\ M\ \rangle \!\ \in$L$_{halte}$.
  \end{itemize}
\end{enumerate}

Donc, on a bien $\langle\ M\ \rangle \!\ \in$L$_{halte}\Leftrightarrow$ f(w)$\in$D.

Étant donné que L$_{halte}$ n'est pas décidable, la réduction prouve que D n'est pas décidable non plus.

\subsubsection{Question 2}

Langage E$\times$F, avec:
\begin{itemize}
  \item E=\{$\langle\ M\ \rangle \!\ $| b$\in$L(M)\}
  \item F=\{$\langle\ M\ \rangle \!\ $| a$\in$L(m) ou b$\in$L(M)\}
\end{itemize}

E$\times$F est une paire de mots tel que le premier mot est dans E, et le second mot est dans F.

On va construire une réducion afin de prouver que ce langage n'est pas décidable.

On prends M$_F$, la machine qui accepte tout, afin d'artificiellement oublier le langage F.\\
On ignore la seconde coordonnée car elle accepte tout le temps.

Soit f, la fonction telle que f(M)=$\begin{pmatrix}\langle\ M'\ \rangle \!\ \\ \langle\ M_F\ \rangle \!\ \end{pmatrix}$,\\
où M' est la machine qui efface b et lance M, et accepte si M s'arrête.

\begin{enumerate}
  \item Montrer que $\langle\ M\ \rangle \!\ \in$L$_{halte}\Rightarrow$f($\langle\ M\ \rangle \!\ $)$\in$E$\times$F.
  \begin{itemize}
    \item Soit $\langle\ M\ \rangle \!\ \in$L$_{halte}$ et $\begin{pmatrix}\langle\ M'\ \rangle \!\ \\ \langle\ M_F\ \rangle \!\ \end{pmatrix}$=
    f($\langle\ M\ \rangle \!\ $)
    \item Si on lance M' sur l'entrée b, alors elle lance M sur l'entrée vide, et M s'arrête, donc M' accepte, donc M'$\in$E.
    \item Donc, f($\langle\ M\ \rangle \!\ $)$\in$E$\times$F.
  \end{itemize}
  \item Montrer que f($\langle\ M\ \rangle \!\ $)$\in$E$\times$F$\Rightarrow$$\langle\ M\ \rangle \!\ \in$L$_{halte}$
  \begin{itemize}
    \item Soit $\langle\ M\ \rangle \!\ $ tel que f($\langle\ M\ \rangle \!\ $)$\in$E$\times$F.
    \item $\Rightarrow$f($\langle\ M\ \rangle \!\ $)=$\begin{pmatrix}\langle\ M'\ \rangle \!\ \\ \langle\ M_F\ \rangle \!\ \end{pmatrix}$
    tel que $\langle\ M'\ \rangle \!\ $)$\in$E.
    \item Donc, $\langle\ M'\ \rangle \!\ $) accepte b, or, sur b, $\langle\ M'\ \rangle \!\ $) efface b et lance M sur le vide, et accepte si et seulement si
    M s'arrête sur l'entrée vide.
    \item Or, vu que M' accepte, M s'arrête sur l'entrée vide.
    \item Donc, $\langle\ M\ \rangle \!\ \in$L$_{halte}$.
  \end{itemize}
\end{enumerate}

Donc, on a bien une réduction de L$_{halte}$ vers E$\times$F, et on sait que L$_{halte}$ n'est pas décidable.\\
Par conséquent, E$\times$F n'est pas décidable.

\subsubsection{Question 5}

$\forall$ MT M, L$_M$=\{w|w$\in$L(M)\} est RE.

L$_M$=\{w| w est accepté par M\} donc, L$_M$ est RE.

\subsubsection{Question 7}

I=\{$\langle\ M\ \rangle \!\ $|$\langle\ M\ \rangle \!\ $ < 2$^{2^{1024}}$ et L(M)={a}\}

À travers le 2$^{2^{1024}}$, on limite le nombre de machines, ce qui crée un langage fini, car il a un nombre de machines finies.\\
Or, un langage fini est forcément décidable.\\
Par conséquent, ce langage est décidable.
